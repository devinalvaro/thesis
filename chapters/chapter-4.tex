\chapter{Implementasi dan Pengujian} \label{ImplementationAndEvaluation}

\section{Implementasi}

Pada bagian ini dijelaskan mengenai implementasi peningkatan kinerja \en{memory swapping} di TensorFlow menurut \nameref{SolutionDesign}. Implementasi yang dilakukan berupa modifikasi terhadap \en{kernel} TensorFlow, terutama di \en{file} \code{tensorflow/core/kernels/\\stack.cc}. Karena modifikasi dilakukan di level \en{kernel}, peningkatan kinerja ini transparan dari sudut pandang \en{end-user} TensorFlow.

Secara garis besar, implementasi dapat dibagi menjadi 2 bagian, yaitu modifikasi \en{swapping out} menurut Kode~\ref{lst:SwapOutPseudocodeOptimized} dan \en{swapping in} menurut Kode~\ref{lst:SwapInPseudocodeOptimized}.

\subsection{Modifikasi \en{Swapping Out}}

Pada Kode~\ref{lst:SwapOutPseudocodeOptimized} yang menjelaskan peningkatan kinerja \en{swapping out}, terdapat metode \code{get\_\\oldest\_unswapped\_tensor\_id} yang berfungsi mengembalikan indeks dari \en{oldest tensor} yang belum di-\en{swap}. Dari sini, \code{Stack} perlu dimodifikasi untuk mengakomodasi metode tersebut.

Pertama, dibutuhkan sebuah \en{queue} untuk menyimpan indeks-indeks \en{tensor} yang belum di-\en{swap}. \en{Queue} dipilih agar indeks \en{tensor} yang di-\en{dequeue} adalah yang terawal (\en{oldest}). Selanjutnya, metode \code{Push} pada \code{Stack} juga perlu dimodifikasi agar ketika sebuah \en{tensor} di-\en{push} ke \en{stack}, indeksnya juga di-\en{enqueue} ke \en{queue} tersebut. Kode~\ref{lst:UnswappedTensorIds} menunjukkan deklarasi \en{queue} dan modifikasi \code{Push} tersebut.

\begin{lstlisting}[language=C++, caption=Deklarasi \code{unswapped\_ids\_} dan modifikasi \code{Push}, label={lst:UnswappedTensorIds}]
class Stack {
  private:
    // ...
    std::queue<int> unswapped_ids_;

  public:
    // ...
    Status Push(const TensorAndAllocation& value) {
      // ...
      stack_.push_back(value);
      int index = stack_.size() - 1;
      unswapped_ids_.push_back(index);
      return Status::OK();
  }
}
\end{lstlisting}

Setelahnya, dapat diimplementasi metode yang mengambil indeks \en{oldest tensor} yaitu \code{GetTensorToSwapOut}. Metode tersebut mengambil indeks \en{oldest tensor} dari \code{unswapped\_ids\_} kemudian meng-\en{assign} \en{tensor} tersebut ke parameter \en{output} \code{value}. Selain itu, indeks juga di-\en{push} ke \code{swapped\_ids\_} yang akan dijelaskan kemudian. Kode~\ref{lst:GetTensorToSwapOut} merupakan simplifikasi dari implementasi \code{GetTensorToSwapOut}.

\begin{lstlisting}[language=C++, caption=Definisi \code{GetTensorToSwapOut}, label={lst:GetTensorToSwapOut}]
class Stack {
  public:
    // ...
    void GetTensorToSwapOut(TensorAndAllocation** value) {
      if (unswapped_ids_.empty())
        return;

      int index = unswapped_ids_.front();
      unswapped_ids_.pop();
      swapped_ids_.push_back(index);
      *value = &stack_[index];
    }
}
\end{lstlisting}

Berikutnya adalah modifikasi terhadap \code{StackPushOp} mengikuti Kode~\ref{lst:SwapOutPseudocodeOptimized}. Seperti pada \en{pseudocode} tersebut, modifikasi tetap mengikuti heuristik yang ada, yaitu melakukan \en{swapping} ketika memori GPU 70\% penuh, namun yang di-\en{swap out} bukanlah \en{tensor} sekarang, malainkan \en{oldest tensor} yang didapat dari \code{GetTensorToSwapOut}. Implementasi modifikasi ini ditunjukkan oleh Kode~\ref{lst:StackPushOp} dengan beberapa penyederhanaan.

\begin{lstlisting}[language=C++, caption=Modifikasi \code{StackPushOp}, label={lst:StackPushOp}]
void StackPushOp::ComputeAsync(OpKernelContext* ctx, DoneCallback done) {
  // ...

  // Push current tensor to Stack.
  ctx, stack->Push({tensor, alloc_attrs, false});
  ctx->set_output(0, tensor);

  // ...

  // If device memory <= 70% full no need to swap.
  // ...
  static constexpr double kOccupancy = 0.7;
  if (stats.bytes_in_use <= (stats.bytes_limit * kOccupancy)) {
    done();
    return;
  }

  // Obtain the oldest unswapped TensorAndAllocation pointer from queue.
  Stack::TensorAndAllocation* oldest_tensor;
  stack->GetTensorToSwapOut(&oldest_tensor);

  // Asynchronously swap out the oldest tensor.
  // ...
  Tensor* cpu_tensor = ...
  device_ctxt->CopyDeviceTensorToCPU(
      &(oldest_tensor->tensor), "StackPush", device, cpu_tensor,
      [stack, oldest_tensor, cpu_tensor, done](const Status& s) {
        if (s.ok()) {
          oldest_tensor->tensor = *cpu_tensor;
          oldest_tensor->swapped_to_cpu = true;
        }
        done();
        delete cpu_tensor;
      });
}
\end{lstlisting}

Pada bagian terakhir Kode~\ref{lst:StackPushOp} di atas terlihat proses \en{swapping out} yang dilakukan terhadap \en{oldest tensor}. \en{Swapping out} dilakukan secara \en{asynchronous} yang setelah selesai dijalankan memanggil sebuah \en{callback function}. \en{Callback} ini mengubah elemen \en{oldest tensor} di \en{stack}, dari \code{tensor} yang merujuk ke \code{device\_tensor} menjadi \code{cpu\_tensor} dan \code{swapped\_to\_cpu} menjadi \code{true} agar pada \en{swapping in} dapat diketahui bahwa \en{tensor} bersangkutan telah di-\en{swap}. (Lebih detail mengenai \code{tensor} dijelaskan pada Bagian~\ref{Tensor}.)

Demikian implementasi pada bagian \en{swapping out}, di mana \en{oldest tensor} di-\en{swap out} agar pada \en{backpropagation} dapat di-\en{swap in} bersamaan dengan berjalannya komputasi, sehingga meningkatkan \en{asynchronicity}. Berikutnya adalah penjelasan mengenai implementasi pada bagian \en{swapping in}.

\subsection{Modifikasi \en{Swapping In}} \label{SwappingInModification}

Pada bagian ini dijelaskan implementasi menurut Kode~\ref{lst:SwapInPseudocodeOptimized} yang menjelaskan peningkatan kinerja pada bagian \en{swapping in}. Namun, sebelum masuk ke bagian tersebut perlu ditunjukkan kode-kode yang mendukung implementasi tersebut.

Sebelumnya, pada Kode~\ref{lst:GetTensorToSwapOut}, indeks di-\en{push} ke \code{swapped\_ids\_}, yaitu sebuah \en{stack} yang menyimpan indeks-indeks \en{tensor} yang di-\en{swap}. Digunakan \en{stack} karena berkebalikan dari \en{swapping out}, \en{swapping in} dilakukan terhadap \en{swapped tensor} yang terbaru (\en{most recent}) dahulu. Kode~\ref{lst:SwappedTensorIds} menunjukkan deklarasi \en{swapped\_ids\_}.

\begin{lstlisting}[language=C++, caption=Deklarasi \code{swapped\_ids\_}, label={lst:SwappedTensorIds}]
class Stack {
  private:
    // ...
    std::queue<int> swapped_ids_;
}
\end{lstlisting}

Selanjutnya adalah penjelasan mengenai \code{GetTensorToSwapIn}, metode yang mengambil indeks \en{tensor} selanjutnya harus di-\en{code}. Secara garis besar, metode ini mirip dengan \code{GetTensorToSwapOut} namun terhadap \code{swapped\_ids\_}, seperti ditunjukkan oleh Kode~\ref{lst:GetTensorToSwapIn}

\begin{lstlisting}[language=C++, caption=Definisi \code{GetTensorToSwapIn}, label={lst:GetTensorToSwapIn}]
class Stack {
  public:
    // ...
    void GetTensorToSwapIn(TensorAndAllocation** value) {
      if (swapped_ids_.empty())
        return;

      int index = swapped_ids_.top();
      swapped_ids_.pop();
      *value = &stack_[index];
    }
}
\end{lstlisting}

Berikutnya adalah modifikasi terhadap \code{StackPopOp} mengikuti Kode~\ref{lst:SwapInPseudocodeOptimized}. Seperti pada \en{pseudocode} tersebut, \code{StackPopOp} dimodifikasi sehingga setelah mem-\en{pop} \en{tensor} teratas dari \en{stack}, dilakukan juga \en{swapping in} terhadap \en{tensor}-\en{tensor} berikutnya secara \en{asynchronous} untuk meningkatkan \en{overlapping} antara komputasi dengan \en{memory transfer}. Implementasi modifikasi ini ditunjukkan oleh Kode~\ref{lst:StackPopOp} dengan beberapa penyederhanaan.

\begin{lstlisting}[language=C++, caption=Modifikasi \code{StackPopOp}, label={lst:StackPopOp}]
void StackPopOp::ComputeAsync(OpKernelContext* ctx, DoneCallback done) {
  // ...

  // Pop a tensor from Stack.
  Stack::TensorAndAllocation value;
  OP_REQUIRES_OK_ASYNC(ctx, stack->Pop(&value), done);

  // If the tensor was swapped out, then swap in (not shown for clarity).
  // ...

  // Asynchronously swap "future" tensors.

  // If device memory still > 90% full don't swap in yet.
  // ...
  static constexpr double kOccupancy = 0.7;
  if (stats.bytes_in_use > (stats.bytes_limit * kOccupancy)) {
    return;
  }

  Stack::TensorAndAllocation* swapped_tensor;
  stack->GetTensorToSwapIn(&swapped_tensor);

  Tensor* device_tensor = ...
  device_ctxt->CopyCPUTensorToDevice(
      &(swapped_tensor->tensor), device, device_tensor,
      [stack, swapped_tensor, device_tensor, index](const Status& s) {
        if (s.ok()) {
          swapped_tensor->tensor = *device_tensor;
          swapped_tensor->swapped_to_cpu = false;
        }
        delete device_tensor;
      });
}
\end{lstlisting}

Pada bagian Kode~\ref{lst:StackPopOp} di atas terlihat proses \en{swapping in} terhadap "\en{future tensors}", yaitu \en{swapped tensors} yang belum akan digunakan untuk komputasi, dengan tujuan meningkatkan \en{overlapping} antara komputasi dengan \en{memory transfer}, seperti yang telah disebutkan.

Terdapat satu lagi hal yang perlu dibahas, yaitu heurustik pada \en{swapping in}. Seperti heuristik pada \en{swapping out}, \en{swapping in} dilakukan ketika memori GPU hampir penuh. Dipilih nilai yang \code{kOccupancy} yang sama yaitu 0.7, untuk menyamai nilai yang digunakan di \en{swapping out}, karena nilai heuristik tersebut sudah ada di versi bawaan TensorFlow.

Demikian penjelasan mengenai implementasi peningkatan kinerja \en{memory swapping} di TensorFlow. Berikutnya dilakukan pengujian terhadap peningkatan kinerja ini dibandingkan dengan versi bawaan TensorFlow.

\section{Pengujian}

Pada bagian ini dijelaskan mengenai pengujian terhadap peningkatan kinerja \en{memory swapping} di TensorFlow yang telah diimplementasikan. Berikut dijelaskan lingkungan pengujian, metode pengujian, dan hasil pengujian dalam berbagai kasus pengujian.

\subsection{Lingkungan Pengujian}

Tabel~\ref{tbl:Specification} memaparkan spesifikasi lingkungan yang digunakan penulis dalam melakukan pengujian.

\begin{table}[htp]
\centering
\caption{Spesifikasi Lingkungan Pengujian}
\label{tbl:Specification}
\small
\begin{tabular}{|l|l|}
\hline
\multicolumn{1}{|c|}{\textbf{Original}} & \multicolumn{1}{c|}{\textbf{Optimized}} \\ \hline
\en{Operating System}                   & Ubuntu 18.04 64-bit                                                                                            \\ \hline
\en{Processor}                          & Intel Xeon CPU @ 2.20 GHz                                                                                       \\ \hline
\en{Memory}                             & 12 GB                                                                                                          \\ \hline
\en{GPU}                                & \begin{tabular}[c]{@{}l@{}}NVIDIA Tesla P4\\ \en{Memory}: 8 GB\\ \en{Driver}: 418.56\\ CUDA: 10.0\end{tabular} \\ \hline
\end{tabular}
\end{table}

\subsection{Metode Pengujian} \label{ExperimentMethod}

Setiap pengujian dilakukan dengan beberapa kali melakukan \en{training} suatu model RNN menggunakan TensorFlow versi asli dan versi \en{improved}, kemudian durasi \en{training} kedua perlakuan dibandingkan dan dihitung rataannya. Selain itu, dilakukan uji statistik lain seperti \en{t-test} untuk memastikan bahwa durasi \en{training} kedua perlakuan memang berbeda.

Model yang digunakan untuk pengujian adalah Char-RNN berdasarkan \citep{karpathy2015unreasonable}, namun telah diadaptasi ke TensorFlow. Secara singkat, char-rnn adalah model RNN yang menerima sekuens teks kemudian membangkitkan sekuens teks lanjutannya, karakter per karakter.

Implementasi model tersebut dengan TensorFlow menggunakan kelas \code{dynamic\_rnn} (disebutkan di Bagian~\ref{WhyRNN}) dengan parameter \code{swap\_memory=True} untuk mengijinkan TensorFlow melakukan \en{memory swapping}. Selain itu, parameter-parameter yang digunakan dalam menjalankan pengujian dipilih sedemikian rupa agar ukuran memori yang dibutuhkan melebihi ukuran memori yang ada, sehingga TensorFlow terpaksa melakukan \en{memory swapping}.

Terdapat dua \en{dataset} yang digunakan untuk pengujian. Yang pertama adalah teks berisi 291.268 karakter (285 KB), setelahnya disebut data kecil. Yang kedua adalah teks berisi 4.573.338 karakter (4.4 MB), setelahnya disebut data besar. Kedua data yang berbeda ukuran ini dipilih untuk membandingkan kinerja peningkatan kinerja terhadap data dalam berbagai ukuran. Selain itu, isi kedua data juga berbeda (data kecil bukan \en{subset} dari data besar).

Pengujian model dilakukan dengan berbagai variasi nilai parameter, yaitu jumlah \en{timestep}, ukuran \en{batch}, dan jumlah \en{epoch}. Pengujian terhadap berbagai jumlah \en{timestep} dilakukan karena jumlah \en{timestep} mempengaruhi ukuran \en{stack}, seperti yang dijelaskan di Bagian~\ref{Stack}. Ukuran \en{batch} pada dasarnya menyesuaikan jumlah \en{timestep}, agar kebutuhan memori TensorFlow relatif konsisten dari kasus ke kasus. Terakhir, variasi jumlah \en{epoch} bertujuan menguji \en{training} dalam berbagai durasi, karena jumlah \en{epoch} secara langsung mempengaruhi durasi \en{training} model. Pada semua kasus, model memiliki 8 \en{hidden layer} yang masing-masing memiliki 512 neuron.

Tabel~\ref{tbl:Cases} menunjukkan kode setiap kasus pengujian beserta parameter-parameter dan \en{dataset} yang digunakan. Kasus-kasus dengan prefiks "A" menggunakan data kecil sedangkan data besar digunakan kasus-kasus berprefiks "B". Kasus-kasus berprefiks sama dibagi lebih lanjut menjadi 3 grup. Grup pertama bersufiks 1-3, dengan jumlah \en{timestep} 25, grup kedua bersufiks 4-6 dengan jumlah \en{timestep} 50, dan grup ketiga bersufiks 7-9, dengan jumlah \en{timestep} 75. Pada setiap grup tersebut, semakin tinggi angka sufiks berarti semakin tinggi jumlah \en{epoch}-nya, yaitu antara 10, 20, dan 40.

\begin{table}[htp]
\centering
\caption{Kode, parameter, dan \en{dataset} setiap kasus pengujian}
\label{tbl:Cases}
\small
\begin{tabular}{|l|l|l|l|l|}
\hline
\textbf{Kode Pengujian} & \textbf{Jumlah \en{Timestep}} & \textbf{Ukuran \en{Batch}} & \textbf{Jumlah \en{Epoch}} & \textbf{\en{Dataset}} \\ \hline
A1            & 25                       & 4125                  & 10                    & Kecil            \\ \hline
A2            & 25                       & 4125                  & 20                    & Kecil            \\ \hline
A3            & 25                       & 4125                  & 40                    & Kecil            \\ \hline
A4            & 50                       & 3000                  & 10                    & Kecil            \\ \hline
A5            & 50                       & 3000                  & 20                    & Kecil            \\ \hline
A6            & 50                       & 3000                  & 40                    & Kecil            \\ \hline
A7            & 75                       & 2500                  & 10                    & Kecil            \\ \hline
A8            & 75                       & 2500                  & 20                    & Kecil            \\ \hline
A9            & 75                       & 2500                  & 40                    & Kecil            \\ \hline
B1            & 25                       & 4125                  & 10                    & Besar            \\ \hline
B2            & 25                       & 4125                  & 20                    & Besar            \\ \hline
B3            & 25                       & 4125                  & 40                    & Besar            \\ \hline
B4            & 50                       & 3000                  & 10                    & Besar            \\ \hline
B5            & 50                       & 3000                  & 20                    & Besar            \\ \hline
B6            & 50                       & 3000                  & 40                    & Besar            \\ \hline
B7            & 75                       & 2500                  & 10                    & Besar            \\ \hline
B8            & 75                       & 2500                  & 20                    & Besar            \\ \hline
B9            & 75                       & 2500                  & 40                    & Besar            \\ \hline
\end{tabular}
\end{table}

\subsection{Hasil Pengujian}

Bagian ini berisi hasil setiap kasus pengujian, diikuti dengan pembahasan hasil pengujian pada bagian selanjutnya. Pada setiap pengujian, ditampilkan durasi \en{training} (dalam satuan detik) kedua perlakuan (TensorFlow asli dan \en{improved}) pada masing-masing 10 kali percobaan.

\subsubsection{Kasus A1}

Tabel~\ref{tbl:A1} menunjukkan hasil pengujian kasus A1. Dengan \en{t-test} didapat \en{p-value} 0.86, sehingga tidak dapat dikatakan bahwa versi \en{improved} berbeda secara signifikan dengan versi asli.

\begin{table}[htp]
\centering
\caption{Hasil pengujian kasus A1}
\label{tbl:A1}
\small
\begin{tabular}{cll}
\hline
\multicolumn{1}{|c|}{\textbf{i}}  & \multicolumn{1}{c|}{\en{\textbf{Original}}} & \multicolumn{1}{c|}{\en{\textbf{Optimized}}} \\ \hline
\multicolumn{1}{|c|}{1}           & \multicolumn{1}{l|}{33.225433349609375}         & \multicolumn{1}{l|}{33.543959617614746}          \\ \hline
\multicolumn{1}{|c|}{2}           & \multicolumn{1}{l|}{33.300493478775024}         & \multicolumn{1}{l|}{33.364500284194946}          \\ \hline
\multicolumn{1}{|c|}{3}           & \multicolumn{1}{l|}{34.74644994735718}          & \multicolumn{1}{l|}{33.57258701324463}           \\ \hline
\multicolumn{1}{|c|}{4}           & \multicolumn{1}{l|}{33.64357852935791}          & \multicolumn{1}{l|}{34.97247886657715}           \\ \hline
\multicolumn{1}{|c|}{5}           & \multicolumn{1}{l|}{33.51314330101013}          & \multicolumn{1}{l|}{33.11316227912903}           \\ \hline
\multicolumn{1}{|c|}{6}           & \multicolumn{1}{l|}{33.47014403343201}          & \multicolumn{1}{l|}{33.4130334854126}            \\ \hline
\multicolumn{1}{|c|}{7}           & \multicolumn{1}{l|}{33.366349935531616}         & \multicolumn{1}{l|}{33.458592891693115}          \\ \hline
\multicolumn{1}{|c|}{8}           & \multicolumn{1}{l|}{33.29462194442749}          & \multicolumn{1}{l|}{33.58814311027527}           \\ \hline
\multicolumn{1}{|c|}{9}           & \multicolumn{1}{l|}{33.35132122039795}          & \multicolumn{1}{l|}{33.24147891998291}           \\ \hline
\multicolumn{1}{|c|}{10}          & \multicolumn{1}{l|}{33.34075593948364}          & \multicolumn{1}{l|}{33.37231993675232}           \\ \hline
\multicolumn{1}{l}{\textbf{Mean}} & 33.52523                                        & 33.56403
\end{tabular}
\end{table}

\subsubsection{Kasus A2}

Tabel~\ref{tbl:A2} menunjukkan hasil pengujian kasus A2. Meski versi \en{improved} memiliki rataan sedikit lebih rendah, \en{p-value} dari \en{t-test} masih 0.14, sehingga sama seperti kasus sebelumnya, tidak dapat dikatakan bahwa versi \en{improved} berbeda secara signifikan dengan versi asli.

\begin{table}[htp]
\centering
\caption{Hasil pengujian kasus A2}
\label{tbl:A2}
\small
\begin{tabular}{cll}
\hline
\multicolumn{1}{|c|}{\textbf{i}}  & \multicolumn{1}{c|}{\en{\textbf{Original}}} & \multicolumn{1}{c|}{\en{\textbf{Optimized}}} \\ \hline
\multicolumn{1}{|c|}{1}           & \multicolumn{1}{l|}{66.12863492965698}          & \multicolumn{1}{l|}{64.3571457862854}            \\ \hline
\multicolumn{1}{|c|}{2}           & \multicolumn{1}{l|}{64.54400324821472}          & \multicolumn{1}{l|}{64.28385186195374}           \\ \hline
\multicolumn{1}{|c|}{3}           & \multicolumn{1}{l|}{64.79296922683716}          & \multicolumn{1}{l|}{64.79301166534424}           \\ \hline
\multicolumn{1}{|c|}{4}           & \multicolumn{1}{l|}{65.09491682052612}          & \multicolumn{1}{l|}{64.81330513954163}           \\ \hline
\multicolumn{1}{|c|}{5}           & \multicolumn{1}{l|}{64.75147080421448}          & \multicolumn{1}{l|}{65.87236380577087}           \\ \hline
\multicolumn{1}{|c|}{6}           & \multicolumn{1}{l|}{65.28346109390259}          & \multicolumn{1}{l|}{64.35014033317566}           \\ \hline
\multicolumn{1}{|c|}{7}           & \multicolumn{1}{l|}{65.07980608940125}          & \multicolumn{1}{l|}{65.6430492401123}            \\ \hline
\multicolumn{1}{|c|}{8}           & \multicolumn{1}{l|}{66.45581388473511}          & \multicolumn{1}{l|}{64.49136257171631}           \\ \hline
\multicolumn{1}{|c|}{9}           & \multicolumn{1}{l|}{64.73718810081482}          & \multicolumn{1}{l|}{64.37896466255188}           \\ \hline
\multicolumn{1}{|c|}{10}          & \multicolumn{1}{l|}{64.67400527000427}          & \multicolumn{1}{l|}{64.30785369873047}           \\ \hline
\multicolumn{1}{l}{\textbf{Mean}} & 65.15423                                        & 64.7291
\end{tabular}
\end{table}

\subsubsection{Kasus A3}

Tabel~\ref{tbl:A3} menunjukkan hasil pengujian kasus A3. Meski versi \en{improved} memiliki rataan sedikit lebih rendah, \en{p-value} dari \en{t-test} masih 0.14, sehingga sama seperti kasus sebelumnya, tidak dapat dikatakan bahwa versi \en{improved} berbeda secara signifikan dengan versi asli.

\begin{table}[htp]
\centering
\caption{Hasil pengujian kasus A3}
\label{tbl:A3}
\small
\begin{tabular}{cll}
\hline
\multicolumn{1}{|c|}{\textbf{i}}  & \multicolumn{1}{c|}{\en{\textbf{Original}}} & \multicolumn{1}{c|}{\en{\textbf{Optimized}}} \\ \hline
\multicolumn{1}{|c|}{1}           & \multicolumn{1}{l|}{127.28371238708496}         & \multicolumn{1}{l|}{126.28772187232971}          \\ \hline
\multicolumn{1}{|c|}{2}           & \multicolumn{1}{l|}{128.12986397743225}         & \multicolumn{1}{l|}{127.92953848838806}          \\ \hline
\multicolumn{1}{|c|}{3}           & \multicolumn{1}{l|}{127.42624640464783}         & \multicolumn{1}{l|}{128.2653489112854}           \\ \hline
\multicolumn{1}{|c|}{4}           & \multicolumn{1}{l|}{127.49524307250977}         & \multicolumn{1}{l|}{126.79777836799622}          \\ \hline
\multicolumn{1}{|c|}{5}           & \multicolumn{1}{l|}{127.46012616157532}         & \multicolumn{1}{l|}{126.79208564758301}          \\ \hline
\multicolumn{1}{|c|}{6}           & \multicolumn{1}{l|}{128.38941287994385}         & \multicolumn{1}{l|}{126.9858250617981}           \\ \hline
\multicolumn{1}{|c|}{7}           & \multicolumn{1}{l|}{128.4031105041504}          & \multicolumn{1}{l|}{128.2889633178711}           \\ \hline
\multicolumn{1}{|c|}{8}           & \multicolumn{1}{l|}{126.78421878814697}         & \multicolumn{1}{l|}{127.4227511882782}           \\ \hline
\multicolumn{1}{|c|}{9}           & \multicolumn{1}{l|}{127.52276182174683}         & \multicolumn{1}{l|}{126.64501905441284}          \\ \hline
\multicolumn{1}{|c|}{10}          & \multicolumn{1}{l|}{127.322434425354}           & \multicolumn{1}{l|}{126.13128423690796}          \\ \hline
\multicolumn{1}{l}{\textbf{Mean}} & 127.62171                                       & 127.15463
\end{tabular}
\end{table}

\subsubsection{Kasus A4}

Tabel~\ref{tbl:A4} menunjukkan hasil pengujian kasus A4. Tidak seperti kasus-kasus sebelumnya, mulai dari kasus ini \en{p-value} dari \en{t-test} sangat kecil yaitu 8.78e-10, sehingga versi \en{improved} dapat dikatakan berbeda secara signifikan dengan versi asli. Selain itu, versi \en{improved} berhasil mengurangi durasi \en{training} sebesar 2.6\%.

\begin{table}[htp]
\centering
\caption{Hasil pengujian kasus A4}
\label{tbl:A4}
\small
\begin{tabular}{cll}
\hline
\multicolumn{1}{|c|}{\textbf{i}}  & \multicolumn{1}{c|}{\en{\textbf{Original}}} & \multicolumn{1}{c|}{\en{\textbf{Optimized}}} \\ \hline
\multicolumn{1}{|c|}{1}           & \multicolumn{1}{l|}{40.13612484931946}          & \multicolumn{1}{l|}{39.17884802818298}           \\ \hline
\multicolumn{1}{|c|}{2}           & \multicolumn{1}{l|}{40.041789054870605}         & \multicolumn{1}{l|}{38.904966831207275}          \\ \hline
\multicolumn{1}{|c|}{3}           & \multicolumn{1}{l|}{40.46419858932495}          & \multicolumn{1}{l|}{38.93816804885864}           \\ \hline
\multicolumn{1}{|c|}{4}           & \multicolumn{1}{l|}{40.2748806476593}           & \multicolumn{1}{l|}{39.33203983306885}           \\ \hline
\multicolumn{1}{|c|}{5}           & \multicolumn{1}{l|}{40.284501791000366}         & \multicolumn{1}{l|}{39.12717938423157}           \\ \hline
\multicolumn{1}{|c|}{6}           & \multicolumn{1}{l|}{40.02491354942322}          & \multicolumn{1}{l|}{39.145731687545776}          \\ \hline
\multicolumn{1}{|c|}{7}           & \multicolumn{1}{l|}{40.04407596588135}          & \multicolumn{1}{l|}{39.27578830718994}           \\ \hline
\multicolumn{1}{|c|}{8}           & \multicolumn{1}{l|}{39.96129322052002}          & \multicolumn{1}{l|}{39.3774037361145}            \\ \hline
\multicolumn{1}{|c|}{9}           & \multicolumn{1}{l|}{40.02649664878845}          & \multicolumn{1}{l|}{38.87396168708801}           \\ \hline
\multicolumn{1}{|c|}{10}          & \multicolumn{1}{l|}{40.079139709472656}         & \multicolumn{1}{l|}{38.73838710784912}           \\ \hline
\multicolumn{1}{l}{\textbf{Mean}} & 40.13374                                        & 39.08925
\end{tabular}
\end{table}

\subsubsection{Kasus A5}

Tabel~\ref{tbl:A5} menunjukkan hasil pengujian kasus A5. Pada kasus ini, versi \en{improved} mengurangi durasi \en{training} lebih jauh dari sebelumnya, yaitu sebesar 2.98\%. Hal ini didukung dengan \en{p-value} dari \en{t-test} yang bernilai 3.73e-16.

\begin{table}[htp]
\centering
\caption{Hasil pengujian kasus A5}
\label{tbl:A5}
\small
\begin{tabular}{cll}
\hline
\multicolumn{1}{|c|}{\textbf{i}}  & \multicolumn{1}{c|}{\en{\textbf{Original}}} & \multicolumn{1}{c|}{\en{\textbf{Optimized}}} \\ \hline
\multicolumn{1}{|c|}{1}           & \multicolumn{1}{l|}{74.41093921661377}          & \multicolumn{1}{l|}{72.1590006351471}            \\ \hline
\multicolumn{1}{|c|}{2}           & \multicolumn{1}{l|}{74.41318488121033}          & \multicolumn{1}{l|}{72.18699049949646}           \\ \hline
\multicolumn{1}{|c|}{3}           & \multicolumn{1}{l|}{74.40296912193298}          & \multicolumn{1}{l|}{72.28742504119873}           \\ \hline
\multicolumn{1}{|c|}{4}           & \multicolumn{1}{l|}{74.37309646606445}          & \multicolumn{1}{l|}{72.21030902862549}           \\ \hline
\multicolumn{1}{|c|}{5}           & \multicolumn{1}{l|}{74.36228227615356}          & \multicolumn{1}{l|}{71.93646121025085}           \\ \hline
\multicolumn{1}{|c|}{6}           & \multicolumn{1}{l|}{74.43094205856323}          & \multicolumn{1}{l|}{72.51015377044678}           \\ \hline
\multicolumn{1}{|c|}{7}           & \multicolumn{1}{l|}{74.56025838851929}          & \multicolumn{1}{l|}{72.26989245414734}           \\ \hline
\multicolumn{1}{|c|}{8}           & \multicolumn{1}{l|}{74.67031264305115}          & \multicolumn{1}{l|}{72.4364082813263}            \\ \hline
\multicolumn{1}{|c|}{9}           & \multicolumn{1}{l|}{74.49252772331238}          & \multicolumn{1}{l|}{72.14049434661865}           \\ \hline
\multicolumn{1}{|c|}{10}          & \multicolumn{1}{l|}{74.56266045570374}          & \multicolumn{1}{l|}{72.33929324150085}           \\ \hline
\multicolumn{1}{l}{\textbf{Mean}} & 74.46792                                        & 72.24764
\end{tabular}
\end{table}

\subsubsection{Kasus A6}

Tabel~\ref{tbl:A6} menunjukkan hasil pengujian kasus A6. Pada kasus ini, pengurangan durasi \en{training} dari versi \en{improved} mengalami penurunan menjadi 2.6\%. Selain itu, \en{p-value} dari \en{t-test} bernilai 8.65e-10.

\begin{table}[htp]
\centering
\caption{Hasil pengujian kasus A6}
\label{tbl:A6}
\small
\begin{tabular}{cll}
\hline
\multicolumn{1}{|c|}{\textbf{i}}  & \multicolumn{1}{c|}{\en{\textbf{Original}}} & \multicolumn{1}{c|}{\en{\textbf{Optimized}}} \\ \hline
\multicolumn{1}{|c|}{1}           & \multicolumn{1}{l|}{140.91950154304504}         & \multicolumn{1}{l|}{138.64457988739014}          \\ \hline
\multicolumn{1}{|c|}{2}           & \multicolumn{1}{l|}{143.09434533119202}         & \multicolumn{1}{l|}{139.24220991134644}          \\ \hline
\multicolumn{1}{|c|}{3}           & \multicolumn{1}{l|}{142.87442326545715}         & \multicolumn{1}{l|}{138.96000599861145}          \\ \hline
\multicolumn{1}{|c|}{4}           & \multicolumn{1}{l|}{143.1363377571106}          & \multicolumn{1}{l|}{139.10242819786072}          \\ \hline
\multicolumn{1}{|c|}{5}           & \multicolumn{1}{l|}{142.86951398849487}         & \multicolumn{1}{l|}{139.42319130897522}          \\ \hline
\multicolumn{1}{|c|}{6}           & \multicolumn{1}{l|}{143.04462718963623}         & \multicolumn{1}{l|}{139.80454635620117}          \\ \hline
\multicolumn{1}{|c|}{7}           & \multicolumn{1}{l|}{143.43231749534607}         & \multicolumn{1}{l|}{138.7737672328949}           \\ \hline
\multicolumn{1}{|c|}{8}           & \multicolumn{1}{l|}{143.14623403549194}         & \multicolumn{1}{l|}{138.92719864845276}          \\ \hline
\multicolumn{1}{|c|}{9}           & \multicolumn{1}{l|}{142.76287984848022}         & \multicolumn{1}{l|}{139.29908204078674}          \\ \hline
\multicolumn{1}{|c|}{10}          & \multicolumn{1}{l|}{142.9169726371765}          & \multicolumn{1}{l|}{138.78625178337097}          \\ \hline
\multicolumn{1}{l}{\textbf{Mean}} & 142.81972                                       & 139.09633
\end{tabular}
\end{table}

\subsubsection{Kasus A7}

Tabel~\ref{tbl:A7} menunjukkan hasil pengujian kasus A7. Pada kasus ini, versi \en{improved} mengurangi durasi \en{training} sebesar 3.13\%. Menurut \en{t-test}, \en{p-value} kasus ini bernilai 1.5e-10.

\begin{table}[htp]
\centering
\caption{Hasil pengujian kasus A7}
\label{tbl:A7}
\small
\begin{tabular}{cll}
\hline
\multicolumn{1}{|c|}{\textbf{i}}  & \multicolumn{1}{c|}{\en{\textbf{Original}}} & \multicolumn{1}{c|}{\en{\textbf{Optimized}}} \\ \hline
\multicolumn{1}{|c|}{1}           & \multicolumn{1}{l|}{50.267709732055664}         & \multicolumn{1}{l|}{48.66734170913696}           \\ \hline
\multicolumn{1}{|c|}{2}           & \multicolumn{1}{l|}{50.27866792678833}          & \multicolumn{1}{l|}{48.92691469192505}           \\ \hline
\multicolumn{1}{|c|}{3}           & \multicolumn{1}{l|}{50.15108680725098}          & \multicolumn{1}{l|}{48.68866848945618}           \\ \hline
\multicolumn{1}{|c|}{4}           & \multicolumn{1}{l|}{50.484827756881714}         & \multicolumn{1}{l|}{49.00160217285156}           \\ \hline
\multicolumn{1}{|c|}{5}           & \multicolumn{1}{l|}{50.29847717285156}          & \multicolumn{1}{l|}{48.76427745819092}           \\ \hline
\multicolumn{1}{|c|}{6}           & \multicolumn{1}{l|}{50.36558985710144}          & \multicolumn{1}{l|}{48.69947695732117}           \\ \hline
\multicolumn{1}{|c|}{7}           & \multicolumn{1}{l|}{50.16042232513428}          & \multicolumn{1}{l|}{48.713934898376465}          \\ \hline
\multicolumn{1}{|c|}{8}           & \multicolumn{1}{l|}{50.38740539550781}          & \multicolumn{1}{l|}{48.90148639678955}           \\ \hline
\multicolumn{1}{|c|}{9}           & \multicolumn{1}{l|}{50.172200202941895}         & \multicolumn{1}{l|}{48.63461136817932}           \\ \hline
\multicolumn{1}{|c|}{10}          & \multicolumn{1}{l|}{50.305076122283936}         & \multicolumn{1}{l|}{48.1265766620636}            \\ \hline
\multicolumn{1}{l}{\textbf{Mean}} & 50.28715                                        & 48.71249
\end{tabular}
\end{table}

\subsubsection{Kasus A8}

Tabel~\ref{tbl:A8} menunjukkan hasil pengujian kasus A8. Pada kasus ini, versi \en{improved} mengurangi durasi \en{training} sebesar 2.63\%. Menurut \en{t-test}, \en{p-value} kasus ini bernilai 2.98e-13.

\begin{table}[htp]
\centering
\caption{Hasil pengujian kasus A8}
\label{tbl:A8}
\small
\begin{tabular}{cll}
\hline
\multicolumn{1}{|c|}{\textbf{i}}  & \multicolumn{1}{c|}{\en{\textbf{Original}}} & \multicolumn{1}{c|}{\en{\textbf{Optimized}}} \\ \hline
\multicolumn{1}{|c|}{1}           & \multicolumn{1}{l|}{95.36041975021362}          & \multicolumn{1}{l|}{92.14931225776672}           \\ \hline
\multicolumn{1}{|c|}{2}           & \multicolumn{1}{l|}{94.47323751449585}          & \multicolumn{1}{l|}{92.51886916160583}           \\ \hline
\multicolumn{1}{|c|}{3}           & \multicolumn{1}{l|}{94.90548205375671}          & \multicolumn{1}{l|}{92.17644190788269}           \\ \hline
\multicolumn{1}{|c|}{4}           & \multicolumn{1}{l|}{94.8862099647522}           & \multicolumn{1}{l|}{92.4691321849823}            \\ \hline
\multicolumn{1}{|c|}{5}           & \multicolumn{1}{l|}{94.68346977233887}          & \multicolumn{1}{l|}{92.04440569877625}           \\ \hline
\multicolumn{1}{|c|}{6}           & \multicolumn{1}{l|}{94.66735601425171}          & \multicolumn{1}{l|}{92.1714437007904}            \\ \hline
\multicolumn{1}{|c|}{7}           & \multicolumn{1}{l|}{95.33400058746338}          & \multicolumn{1}{l|}{92.12393546104431}           \\ \hline
\multicolumn{1}{|c|}{8}           & \multicolumn{1}{l|}{94.41763091087341}          & \multicolumn{1}{l|}{92.80223345756531}           \\ \hline
\multicolumn{1}{|c|}{9}           & \multicolumn{1}{l|}{95.08026027679443}          & \multicolumn{1}{l|}{92.40912294387817}           \\ \hline
\multicolumn{1}{|c|}{10}          & \multicolumn{1}{l|}{94.63182210922241}          & \multicolumn{1}{l|}{92.67259740829468}           \\ \hline
\multicolumn{1}{l}{\textbf{Mean}} & 94.84399                                        & 92.35375
\end{tabular}
\end{table}

\subsubsection{Kasus A9}

Tabel~\ref{tbl:A9} menunjukkan hasil pengujian kasus A9. Pada kasus ini, versi \en{improved} mengurangi durasi \en{training} sebesar 2.84\%. Menurut \en{t-test}, \en{p-value} kasus ini bernilai 3.83e-17.

\begin{table}[htp]
\centering
\caption{Hasil pengujian kasus A9}
\label{tbl:A9}
\small
\begin{tabular}{cll}
\hline
\multicolumn{1}{|c|}{\textbf{i}}  & \multicolumn{1}{c|}{\en{\textbf{Original}}} & \multicolumn{1}{c|}{\en{\textbf{Optimized}}} \\ \hline
\multicolumn{1}{|c|}{1}           & \multicolumn{1}{l|}{184.15153622627258}         & \multicolumn{1}{l|}{179.35685920715332}          \\ \hline
\multicolumn{1}{|c|}{2}           & \multicolumn{1}{l|}{183.87400341033936}         & \multicolumn{1}{l|}{178.5384087562561}           \\ \hline
\multicolumn{1}{|c|}{3}           & \multicolumn{1}{l|}{184.23513293266296}         & \multicolumn{1}{l|}{178.9195113182068}           \\ \hline
\multicolumn{1}{|c|}{4}           & \multicolumn{1}{l|}{184.94609141349792}         & \multicolumn{1}{l|}{178.95625042915344}          \\ \hline
\multicolumn{1}{|c|}{5}           & \multicolumn{1}{l|}{184.47301125526428}         & \multicolumn{1}{l|}{178.88424515724182}          \\ \hline
\multicolumn{1}{|c|}{6}           & \multicolumn{1}{l|}{184.2485635280609}          & \multicolumn{1}{l|}{179.2267963886261}           \\ \hline
\multicolumn{1}{|c|}{7}           & \multicolumn{1}{l|}{184.3339183330536}          & \multicolumn{1}{l|}{179.6547224521637}           \\ \hline
\multicolumn{1}{|c|}{8}           & \multicolumn{1}{l|}{184.79059147834778}         & \multicolumn{1}{l|}{178.65706658363342}          \\ \hline
\multicolumn{1}{|c|}{9}           & \multicolumn{1}{l|}{183.53205037117004}         & \multicolumn{1}{l|}{178.99056673049927}          \\ \hline
\multicolumn{1}{|c|}{10}          & \multicolumn{1}{l|}{184.18417716026306}         & \multicolumn{1}{l|}{179.2834174633026}           \\ \hline
\multicolumn{1}{l}{\textbf{Mean}} & 184.27691                                       & 179.04678
\end{tabular}
\end{table}

\subsubsection{Kasus B1}

Tabel~\ref{tbl:B1} menunjukkan hasil pengujian kasus B1. Dengan \en{t-test} didapat \en{p-value} 0.26, sehingga tidak dapat dikatakan bahwa versi \en{improved} berbeda secara signifikan dengan versi asli.

\begin{table}[htp]
\centering
\caption{Hasil pengujian kasus B1}
\label{tbl:B1}
\small
\begin{tabular}{cll}
\hline
\multicolumn{1}{|c|}{\textbf{i}}  & \multicolumn{1}{c|}{\en{\textbf{Original}}} & \multicolumn{1}{c|}{\en{\textbf{Optimized}}} \\ \hline
\multicolumn{1}{|c|}{1}           & \multicolumn{1}{l|}{420.8520050048828}          & \multicolumn{1}{l|}{419.5176672935486}           \\ \hline
\multicolumn{1}{|c|}{2}           & \multicolumn{1}{l|}{421.1976397037506}          & \multicolumn{1}{l|}{420.0213441848755}           \\ \hline
\multicolumn{1}{|c|}{3}           & \multicolumn{1}{l|}{420.57384419441223}         & \multicolumn{1}{l|}{421.6392922401428}           \\ \hline
\multicolumn{1}{|c|}{4}           & \multicolumn{1}{l|}{421.9959976673126}          & \multicolumn{1}{l|}{421.11664962768555}          \\ \hline
\multicolumn{1}{|c|}{5}           & \multicolumn{1}{l|}{418.56204295158386}         & \multicolumn{1}{l|}{422.30104398727417}          \\ \hline
\multicolumn{1}{|c|}{6}           & \multicolumn{1}{l|}{422.391574382782}           & \multicolumn{1}{l|}{421.1686770915985}           \\ \hline
\multicolumn{1}{|c|}{7}           & \multicolumn{1}{l|}{422.525634765625}           & \multicolumn{1}{l|}{420.85806703567505}          \\ \hline
\multicolumn{1}{|c|}{8}           & \multicolumn{1}{l|}{421.9921326637268}          & \multicolumn{1}{l|}{419.0454249382019}           \\ \hline
\multicolumn{1}{|c|}{9}           & \multicolumn{1}{l|}{420.93663001060486}         & \multicolumn{1}{l|}{421.06646966934204}          \\ \hline
\multicolumn{1}{|c|}{10}          & \multicolumn{1}{l|}{421.6852512359619}          & \multicolumn{1}{l|}{420.33892846107483}          \\ \hline
\multicolumn{1}{l}{\textbf{Mean}} & 421.27128                                       & 420.70736
\end{tabular}
\end{table}

\subsubsection{Kasus B2}

Tabel~\ref{tbl:B2} menunjukkan hasil pengujian kasus B2. Pada kasus ini, versi \en{improved} hanya mengurangi durasi \en{training} sebesar 0.37\%. Meskipun begitu, menurut \en{t-test}, \en{p-value} bernilai 0.0009 sehingga dapat dikatakan kedua perlakuan berbeda secara signifikan.

\begin{table}[htp]
\centering
\caption{Hasil pengujian kasus B2}
\label{tbl:B2}
\small
\begin{tabular}{cll}
\hline
\multicolumn{1}{|c|}{\textbf{i}}  & \multicolumn{1}{c|}{\en{\textbf{Original}}} & \multicolumn{1}{c|}{\en{\textbf{Optimized}}} \\ \hline
\multicolumn{1}{|c|}{1}           & \multicolumn{1}{l|}{841.4169640541077}          & \multicolumn{1}{l|}{834.6424646377563}           \\ \hline
\multicolumn{1}{|c|}{2}           & \multicolumn{1}{l|}{842.8569185733795}          & \multicolumn{1}{l|}{840.5790700912476}           \\ \hline
\multicolumn{1}{|c|}{3}           & \multicolumn{1}{l|}{842.6284821033478}          & \multicolumn{1}{l|}{841.473141670227}            \\ \hline
\multicolumn{1}{|c|}{4}           & \multicolumn{1}{l|}{842.2062222957611}          & \multicolumn{1}{l|}{839.4596951007843}           \\ \hline
\multicolumn{1}{|c|}{5}           & \multicolumn{1}{l|}{842.6023080348969}          & \multicolumn{1}{l|}{841.4130139350891}           \\ \hline
\multicolumn{1}{|c|}{6}           & \multicolumn{1}{l|}{842.0640499591827}          & \multicolumn{1}{l|}{837.088446855545}            \\ \hline
\multicolumn{1}{|c|}{7}           & \multicolumn{1}{l|}{842.4598443508148}          & \multicolumn{1}{l|}{840.2163150310516}           \\ \hline
\multicolumn{1}{|c|}{8}           & \multicolumn{1}{l|}{842.9509062767029}          & \multicolumn{1}{l|}{839.8717386722565}           \\ \hline
\multicolumn{1}{|c|}{9}           & \multicolumn{1}{l|}{842.0658278465271}          & \multicolumn{1}{l|}{838.8177363872528}           \\ \hline
\multicolumn{1}{|c|}{10}          & \multicolumn{1}{l|}{842.4039092063904}          & \multicolumn{1}{l|}{838.4253575801849}           \\ \hline
\multicolumn{1}{l}{\textbf{Mean}} & 842.36554                                       & 839.1987
\end{tabular}
\end{table}

\subsubsection{Kasus B3}

Tabel~\ref{tbl:B3} menunjukkan hasil pengujian kasus B3. Pada kasus ini, versi \en{improved} hanya mengurangi durasi \en{training} sebesar 0.7\%. Meskipun begitu, menurut \en{t-test}, \en{p-value} bernilai 0.0005 sehingga dapat dikatakan kedua perlakuan berbeda secara signifikan.

\begin{table}[htp]
\centering
\caption{Hasil pengujian kasus B3}
\label{tbl:B3}
\small
\begin{tabular}{cll}
\hline
\multicolumn{1}{|c|}{\textbf{i}}  & \multicolumn{1}{c|}{\en{\textbf{Original}}} & \multicolumn{1}{c|}{\en{\textbf{Optimized}}} \\ \hline
\multicolumn{1}{|c|}{1}           & \multicolumn{1}{l|}{1683.7960169315338}         & \multicolumn{1}{l|}{1673.2739737033844}          \\ \hline
\multicolumn{1}{|c|}{2}           & \multicolumn{1}{l|}{1687.297322511673}          & \multicolumn{1}{l|}{1679.99423289299}            \\ \hline
\multicolumn{1}{|c|}{3}           & \multicolumn{1}{l|}{1686.4014761447906}         & \multicolumn{1}{l|}{1671.2808508872986}          \\ \hline
\multicolumn{1}{|c|}{4}           & \multicolumn{1}{l|}{1681.5528180599213}         & \multicolumn{1}{l|}{1669.1223888397217}          \\ \hline
\multicolumn{1}{|c|}{5}           & \multicolumn{1}{l|}{1690.638794183731}          & \multicolumn{1}{l|}{1680.7036063671112}          \\ \hline
\multicolumn{1}{|c|}{6}           & \multicolumn{1}{l|}{1679.570838689804}          & \multicolumn{1}{l|}{1681.4491155147552}          \\ \hline
\multicolumn{1}{|c|}{7}           & \multicolumn{1}{l|}{1687.8227655887604}         & \multicolumn{1}{l|}{1656.0113110542297}          \\ \hline
\multicolumn{1}{|c|}{8}           & \multicolumn{1}{l|}{1681.795575618744}          & \multicolumn{1}{l|}{1676.8696596622467}          \\ \hline
\multicolumn{1}{|c|}{9}           & \multicolumn{1}{l|}{1684.5717134475708}         & \multicolumn{1}{l|}{1674.0421373844147}          \\ \hline
\multicolumn{1}{|c|}{10}          & \multicolumn{1}{l|}{1689.3503987789154}         & \multicolumn{1}{l|}{1671.2718181610107}          \\ \hline
\multicolumn{1}{l}{\textbf{Mean}} & 1685.27977                                      & 1673.40191
\end{tabular}
\end{table}

\subsubsection{Kasus B4}

Tabel~\ref{tbl:B4} menunjukkan hasil pengujian kasus B4. Pada kasus ini, versi \en{improved} mengurangi durasi \en{training} sebesar 3.02\%. Berdasarkan \en{t-test}, nilai \en{p-value} kasus ini adalah 7.73e-19.

\begin{table}[htp]
\centering
\caption{Hasil pengujian kasus B4}
\label{tbl:B4}
\small
\begin{tabular}{cll}
\hline
\multicolumn{1}{|c|}{\textbf{i}}  & \multicolumn{1}{c|}{\en{\textbf{Original}}} & \multicolumn{1}{c|}{\en{\textbf{Optimized}}} \\ \hline
\multicolumn{1}{|c|}{1}           & \multicolumn{1}{l|}{487.29574179649353}         & \multicolumn{1}{l|}{472.343798160553}            \\ \hline
\multicolumn{1}{|c|}{2}           & \multicolumn{1}{l|}{487.1750123500824}          & \multicolumn{1}{l|}{471.7209942340851}           \\ \hline
\multicolumn{1}{|c|}{3}           & \multicolumn{1}{l|}{486.4116704463959}          & \multicolumn{1}{l|}{472.60439133644104}          \\ \hline
\multicolumn{1}{|c|}{4}           & \multicolumn{1}{l|}{487.04327392578125}         & \multicolumn{1}{l|}{474.2424397468567}           \\ \hline
\multicolumn{1}{|c|}{5}           & \multicolumn{1}{l|}{488.5537300109863}          & \multicolumn{1}{l|}{473.7751681804657}           \\ \hline
\multicolumn{1}{|c|}{6}           & \multicolumn{1}{l|}{487.0935254096985}          & \multicolumn{1}{l|}{472.0223639011383}           \\ \hline
\multicolumn{1}{|c|}{7}           & \multicolumn{1}{l|}{487.66953206062317}         & \multicolumn{1}{l|}{472.12717366218567}          \\ \hline
\multicolumn{1}{|c|}{8}           & \multicolumn{1}{l|}{486.98822927474976}         & \multicolumn{1}{l|}{472.8724088668823}           \\ \hline
\multicolumn{1}{|c|}{9}           & \multicolumn{1}{l|}{487.86662673950195}         & \multicolumn{1}{l|}{472.0631272792816}           \\ \hline
\multicolumn{1}{|c|}{10}          & \multicolumn{1}{l|}{487.58010816574097}         & \multicolumn{1}{l|}{472.9280960559845}           \\ \hline
\multicolumn{1}{l}{\textbf{Mean}} & 487.36775                                       & 472.67
\end{tabular}
\end{table}

\subsubsection{Kasus B5}

Tabel~\ref{tbl:B5} menunjukkan hasil pengujian kasus B5. Pada kasus ini, versi \en{improved} mengurangi durasi \en{training} sebesar 2.92\%. Berdasarkan \en{t-test}, nilai \en{p-value} kasus ini adalah 1.03e-21.

\begin{table}[htp]
\centering
\caption{Hasil pengujian kasus B5}
\label{tbl:B5}
\small
\begin{tabular}{cll}
\hline
\multicolumn{1}{|c|}{\textbf{i}}  & \multicolumn{1}{c|}{\en{\textbf{Original}}} & \multicolumn{1}{c|}{\en{\textbf{Optimized}}} \\ \hline
\multicolumn{1}{|c|}{1}           & \multicolumn{1}{l|}{969.5740151405334}          & \multicolumn{1}{l|}{942.8602046966553}           \\ \hline
\multicolumn{1}{|c|}{2}           & \multicolumn{1}{l|}{971.9089589118958}          & \multicolumn{1}{l|}{939.6011493206024}           \\ \hline
\multicolumn{1}{|c|}{3}           & \multicolumn{1}{l|}{968.7835218906403}          & \multicolumn{1}{l|}{939.6865088939667}           \\ \hline
\multicolumn{1}{|c|}{4}           & \multicolumn{1}{l|}{968.95157289505}            & \multicolumn{1}{l|}{941.9455673694611}           \\ \hline
\multicolumn{1}{|c|}{5}           & \multicolumn{1}{l|}{969.3444783687592}          & \multicolumn{1}{l|}{940.6934111118317}           \\ \hline
\multicolumn{1}{|c|}{6}           & \multicolumn{1}{l|}{970.9714493751526}          & \multicolumn{1}{l|}{942.7242720127106}           \\ \hline
\multicolumn{1}{|c|}{7}           & \multicolumn{1}{l|}{969.1799998283386}          & \multicolumn{1}{l|}{940.6590757369995}           \\ \hline
\multicolumn{1}{|c|}{8}           & \multicolumn{1}{l|}{968.3928897380829}          & \multicolumn{1}{l|}{941.939875125885}            \\ \hline
\multicolumn{1}{|c|}{9}           & \multicolumn{1}{l|}{968.9738261699677}          & \multicolumn{1}{l|}{941.7099456787109}           \\ \hline
\multicolumn{1}{|c|}{10}          & \multicolumn{1}{l|}{969.9917275905609}          & \multicolumn{1}{l|}{941.0221922397614}           \\ \hline
\multicolumn{1}{l}{\textbf{Mean}} & 969.60724                                       & 941.28422
\end{tabular}
\end{table}

\subsubsection{Kasus B6}

Tabel~\ref{tbl:B6} menunjukkan hasil pengujian kasus B6. Pada kasus ini, versi \en{improved} mengurangi durasi \en{training} sebesar 2.87\%. Berdasarkan \en{t-test}, nilai \en{p-value} kasus ini adalah 7.24e-16.

\begin{table}[htp]
\centering
\caption{Hasil pengujian kasus B6}
\label{tbl:B6}
\small
\begin{tabular}{cll}
\hline
\multicolumn{1}{|c|}{\textbf{i}}  & \multicolumn{1}{c|}{\en{\textbf{Original}}} & \multicolumn{1}{c|}{\en{\textbf{Optimized}}} \\ \hline
\multicolumn{1}{|c|}{1}           & \multicolumn{1}{l|}{1930.673156261444}          & \multicolumn{1}{l|}{1881.7857682704926}          \\ \hline
\multicolumn{1}{|c|}{2}           & \multicolumn{1}{l|}{1935.568766117096}          & \multicolumn{1}{l|}{1870.272988319397}           \\ \hline
\multicolumn{1}{|c|}{3}           & \multicolumn{1}{l|}{1931.9509241580963}         & \multicolumn{1}{l|}{1875.614033460617}           \\ \hline
\multicolumn{1}{|c|}{4}           & \multicolumn{1}{l|}{1931.528177022934}          & \multicolumn{1}{l|}{1874.8612790107727}          \\ \hline
\multicolumn{1}{|c|}{5}           & \multicolumn{1}{l|}{1930.9445390701294}         & \multicolumn{1}{l|}{1876.0342326164246}          \\ \hline
\multicolumn{1}{|c|}{6}           & \multicolumn{1}{l|}{1931.7676932811737}         & \multicolumn{1}{l|}{1882.3104836940765}          \\ \hline
\multicolumn{1}{|c|}{7}           & \multicolumn{1}{l|}{1931.8324806690216}         & \multicolumn{1}{l|}{1876.8986308574677}          \\ \hline
\multicolumn{1}{|c|}{8}           & \multicolumn{1}{l|}{1927.9467866420746}         & \multicolumn{1}{l|}{1873.3782126903534}          \\ \hline
\multicolumn{1}{|c|}{9}           & \multicolumn{1}{l|}{1935.1201288700104}         & \multicolumn{1}{l|}{1871.9598927497864}          \\ \hline
\multicolumn{1}{|c|}{10}          & \multicolumn{1}{l|}{1929.9509518146515}         & \multicolumn{1}{l|}{1879.004242181778}           \\ \hline
\multicolumn{1}{l}{\textbf{Mean}} & 1931.72836                                      & 1876.21198
\end{tabular}
\end{table}

\subsubsection{Kasus B7}

Tabel~\ref{tbl:B7} menunjukkan hasil pengujian kasus B7. Pada kasus ini, versi \en{improved} mengurangi durasi \en{training} sebesar 2.54\%. Menurut \en{t-test}, \en{p-value} kasus ini bernilai 2.44e-9.

\begin{table}[htp]
\centering
\caption{Hasil pengujian kasus B7}
\label{tbl:B7}
\small
\begin{tabular}{cll}
\hline
\multicolumn{1}{|c|}{\textbf{i}}  & \multicolumn{1}{c|}{\en{\textbf{Original}}} & \multicolumn{1}{c|}{\en{\textbf{Optimized}}} \\ \hline
\multicolumn{1}{|c|}{1}           & \multicolumn{1}{l|}{497.5256621837616}          & \multicolumn{1}{l|}{485.2355446815491}           \\ \hline
\multicolumn{1}{|c|}{2}           & \multicolumn{1}{l|}{496.89878702163696}         & \multicolumn{1}{l|}{483.9988577365875}           \\ \hline
\multicolumn{1}{|c|}{3}           & \multicolumn{1}{l|}{498.9247143268585}          & \multicolumn{1}{l|}{484.1502511501312}           \\ \hline
\multicolumn{1}{|c|}{4}           & \multicolumn{1}{l|}{496.73374104499817}         & \multicolumn{1}{l|}{487.18865609169006}          \\ \hline
\multicolumn{1}{|c|}{5}           & \multicolumn{1}{l|}{499.26302218437195}         & \multicolumn{1}{l|}{483.7039484977722}           \\ \hline
\multicolumn{1}{|c|}{6}           & \multicolumn{1}{l|}{498.6834931373596}          & \multicolumn{1}{l|}{485.97945642471313}          \\ \hline
\multicolumn{1}{|c|}{7}           & \multicolumn{1}{l|}{496.67711067199707}         & \multicolumn{1}{l|}{484.14442205429077}          \\ \hline
\multicolumn{1}{|c|}{8}           & \multicolumn{1}{l|}{497.69861221313477}         & \multicolumn{1}{l|}{487.8638529777527}           \\ \hline
\multicolumn{1}{|c|}{9}           & \multicolumn{1}{l|}{499.24420976638794}         & \multicolumn{1}{l|}{480.5082628726959}           \\ \hline
\multicolumn{1}{|c|}{10}          & \multicolumn{1}{l|}{496.6339108943939}          & \multicolumn{1}{l|}{488.9758915901184}           \\ \hline
\multicolumn{1}{l}{\textbf{Mean}} & 497.82833                                       & 485.17491
\end{tabular}
\end{table}

\subsubsection{Kasus B8}

Tabel~\ref{tbl:B8} menunjukkan hasil pengujian kasus B8. Pada kasus ini, versi \en{improved} mengurangi durasi \en{training} sebesar 2.71\%. Menurut \en{t-test}, \en{p-value} kasus ini bernilai 2.8e-18.

\begin{table}[htp]
\centering
\caption{Hasil pengujian kasus B8}
\label{tbl:B8}
\small
\begin{tabular}{cll}
\hline
\multicolumn{1}{|c|}{\textbf{i}}  & \multicolumn{1}{c|}{\en{\textbf{Original}}} & \multicolumn{1}{c|}{\en{\textbf{Optimized}}} \\ \hline
\multicolumn{1}{|c|}{1}           & \multicolumn{1}{l|}{989.7410943508148}          & \multicolumn{1}{l|}{962.6812777519226}           \\ \hline
\multicolumn{1}{|c|}{2}           & \multicolumn{1}{l|}{990.8226969242096}          & \multicolumn{1}{l|}{963.4447431564331}           \\ \hline
\multicolumn{1}{|c|}{3}           & \multicolumn{1}{l|}{988.5084526538849}          & \multicolumn{1}{l|}{965.2729036808014}           \\ \hline
\multicolumn{1}{|c|}{4}           & \multicolumn{1}{l|}{992.6685571670532}          & \multicolumn{1}{l|}{963.35018658638}             \\ \hline
\multicolumn{1}{|c|}{5}           & \multicolumn{1}{l|}{991.4316079616547}          & \multicolumn{1}{l|}{962.4873623847961}           \\ \hline
\multicolumn{1}{|c|}{6}           & \multicolumn{1}{l|}{993.6044535636902}          & \multicolumn{1}{l|}{964.7575905323029}           \\ \hline
\multicolumn{1}{|c|}{7}           & \multicolumn{1}{l|}{991.5941832065582}          & \multicolumn{1}{l|}{966.828458070755}            \\ \hline
\multicolumn{1}{|c|}{8}           & \multicolumn{1}{l|}{989.194974899292}           & \multicolumn{1}{l|}{965.7361724376678}           \\ \hline
\multicolumn{1}{|c|}{9}           & \multicolumn{1}{l|}{990.9936196804047}          & \multicolumn{1}{l|}{964.3605980873108}           \\ \hline
\multicolumn{1}{|c|}{10}          & \multicolumn{1}{l|}{993.2517943382263}          & \multicolumn{1}{l|}{963.8871579170227}           \\ \hline
\multicolumn{1}{l}{\textbf{Mean}} & 991.18114                                       & 964.28065
\end{tabular}
\end{table}

\subsubsection{Kasus B9}

Tabel~\ref{tbl:B9} menunjukkan hasil pengujian kasus B9. Pada kasus ini, versi \en{improved} mengurangi durasi \en{training} sebesar 2.66\%. Menurut \en{t-test}, \en{p-value} kasus ini bernilai 4.73e-17.

\begin{table}[htp]
\centering
\caption{Hasil pengujian kasus B9}
\label{tbl:B9}
\small
\begin{tabular}{cll}
\hline
\multicolumn{1}{|c|}{\textbf{i}} & \multicolumn{1}{c|}{\en{\textbf{Original}}} & \multicolumn{1}{c|}{\en{\textbf{Optimized}}} \\ \hline
\multicolumn{1}{|c|}{1}          & \multicolumn{1}{l|}{1979.7677886486053}         & \multicolumn{1}{l|}{1924.7396051883698}          \\ \hline
\multicolumn{1}{|c|}{2}          & \multicolumn{1}{l|}{1975.929737329483}          & \multicolumn{1}{l|}{1926.9788799285889}          \\ \hline
\multicolumn{1}{|c|}{3}          & \multicolumn{1}{l|}{1975.862783908844}          & \multicolumn{1}{l|}{1927.6497223377228}          \\ \hline
\multicolumn{1}{|c|}{4}          & \multicolumn{1}{l|}{1977.7059586048126}         & \multicolumn{1}{l|}{1923.6978118419647}          \\ \hline
\multicolumn{1}{|c|}{5}          & \multicolumn{1}{l|}{1977.1068603992462}         & \multicolumn{1}{l|}{1921.677785873413}           \\ \hline
\multicolumn{1}{|c|}{6}          & \multicolumn{1}{l|}{1976.8533494472504}         & \multicolumn{1}{l|}{1922.5078818798065}          \\ \hline
\multicolumn{1}{|c|}{7}          & \multicolumn{1}{l|}{1971.7307133674622}         & \multicolumn{1}{l|}{1929.644939661026}           \\ \hline
\multicolumn{1}{|c|}{8}          & \multicolumn{1}{l|}{1976.1869127750397}         & \multicolumn{1}{l|}{1928.4419906139374}          \\ \hline
\multicolumn{1}{|c|}{9}          & \multicolumn{1}{l|}{1980.8781063556671}         & \multicolumn{1}{l|}{1920.0823383331299}          \\ \hline
\multicolumn{1}{|c|}{10}         & \multicolumn{1}{l|}{1978.7299919128418}         & \multicolumn{1}{l|}{1919.1239686012268}          \\ \hline
\textbf{Mean}                    & 1977.07522                                      & 1924.45449
\end{tabular}
\end{table}

\subsection{Pembahasan Hasil Pengujian}

Tabel~\ref{tbl:Overview} merangkum rataan kedua perlakuan, persentase selisihnya, serta apakah kedua perlakuan berbeda secara signifikan (menurut \en{t-test}) untuk masing-masing kasus pengujian.

\begin{table}[htp]
\centering
\caption{Rangkuman hasil pengujian}
\label{tbl:Overview}
\small
\begin{tabular}{|l|l|l|l|l|}
\hline
\textbf{Kode Pengujian} & \multicolumn{1}{c|}{\textbf{Rataan \en{Original}}} & \multicolumn{1}{c|}{\textbf{Rataan \en{Optimized}}} & \multicolumn{1}{c|}{\textbf{Selisih}} & \multicolumn{1}{c|}{\textbf{Berbeda?}} \\ \hline
A1            & 33.52523                                           & 33.56403                                            & -0.11\%                               & Tidak                                  \\ \hline
A2            & 65.15423                                           & 64.7291                                             & 0.65\%                                & Tidak                                  \\ \hline
A3            & 127.62171                                          & 127.15463                                           & 0.36\%                                & Tidak                                  \\ \hline
A4            & 40.13374                                           & 39.08925                                            & 2.6\%                                 & Ya                                     \\ \hline
A5            & 74.46792                                           & 72.24764                                            & 2.98\%                                & Ya                                     \\ \hline
A6            & 142.81972                                          & 139.09633                                           & 2.6\%                                 & Ya                                     \\ \hline
A7            & 50.28715                                           & 48.71249                                            & 3.13\%                                & Ya                                     \\ \hline
A8            & 94.84399                                           & 92.35375                                            & 2.63\%                                & Ya                                     \\ \hline
A9            & 184.27691                                          & 179.04678                                           & 2.84\%                                & Ya                                     \\ \hline
B1            & 421.27128                                          & 420.70736                                           & 0.13\%                                & Tidak                                  \\ \hline
B2            & 842.36554                                          & 839.1987                                            & 0.37\%                                & Ya                                     \\ \hline
B3            & 1685.27977                                         & 1673.40191                                          & 0.7\%                                 & Ya                                     \\ \hline
B4            & 487.36775                                          & 472.67                                              & 3.02\%                                & Ya                                     \\ \hline
B5            & 969.60724                                          & 941.28422                                           & 2.92\%                                & Ya                                     \\ \hline
B6            & 1931.72836                                         & 1876.21198                                          & 2.87\%                                & Ya                                     \\ \hline
B7            & 497.82833                                          & 485.17491                                           & 2.54\%                                & Ya                                     \\ \hline
B8            & 991.18114                                          & 964.28065                                           & 2.71\%                                & Ya                                     \\ \hline
B9            & 1977.07522                                         & 1924.45449                                          & 2.66\%                                & Ya                                     \\ \hline
\end{tabular}
\end{table}

Pembahasan dilakukan dengan melihat hasil pengujian, terutama bagaimana variabel-variabel seperti jumlah \en{timestep}, jumlah \en{epoch}, dan ukuran \en{dataset} mempengaruhi selisih durasi \en{training} antara kedua perlakuan.

Kedua perlakuan tidak berbeda secara signifikan hanya pada kasus A1, A2, A3, dan B1. Persamaan dari kasus-kasus tersebut adalah jumlah \en{timestep} yang relatif kecil, yaitu 25. Pada semua kasus dengan jumlah \en{timestep} lebih besar, seperti 50 dan 75, kedua perlakuan berbeda secara signifikan. Kesimpulan yang dapat diambil adalah jumlah \en{timestep} perlu melewati batas tertentu agar versi \en{improved} dapat membuat perbedaan yang signifikan.

Pada kasus A1, A2, A3, B1, B2, dan B3, selisih durasi \en{training} kurang dari 0.7\%, sedangkan pada kasus-kasus sisanya selisih konsisten antara 2.5\% hingga 3.2\%. Sama seperti sebelumnya, pada kasus-kasus ini digunakan jumlah \en{timestep} yang relatif kecil, sehingga memang setelah \en{timestep} tertentu barulah versi \en{improved} dapat mengurangi durasi \en{training} secara signifikan.

Penjelasan kedua fenomena tersebut terkait dengan Bagian~\ref{Stack} yang menjelaskan bahwa jumlah \en{timestep} berbanding lurus dengan jumlah elemen \en{stack}. Jumlah elemen \en{stack} yang lebih besar memberikan waktu lebih bagi "\en{future tensors}" untuk di-\en{swap in} secara \en{asynchronous} bersamaan dengan komputasi, seperti yang dijelaskan di Bagian~\ref{SwappingInModification}.

Selanjutnya, pengujian dilakukan terhadap jumlah \en{epoch} yang berbeda-beda untuk mengamati apakah lama durasi \en{training} mempengaruhi selisihnya antara kedua perlakuan. Dari hasil pengujian yang didapat, antara \en{jumlah epoch} yang kecil dan besar, selisih dapat naik (A8 ke A9) maupun turun (A8 ke B6). Maka dari itu, tidak dapat disimpulkan bahwa \en{jumlah epoch} mempengaruhi selisih antara versi asli dan versi \en{improved}.

Kemudian, antara kasus dengan prefiks "A" dan prefiks "B", hanya berbeda pada \en{dataset} yang digunakan, yaitu data kecil dan besar (disebut di Bagian~\ref{ExperimentMethod}). Dengan mengamati hasil pengujian, tidak terlihat pola yang menunjukkan bahwa jenis \en{dataset} mempengaruhi selisih durasi \en{training}. Artinya, versi \en{improved} dapat bekerja untuk berbagai jenis \en{dataset}.

Secara keseluruhan, dari hasil pengujian dapat disimpulkan bahwa selisih durasi \en{training} antara versi asli dan versi \en{improved} paling dipengaruhi oleh jumlah \en{timestep}, sedangkan jumlah \en{epoch} dan jenis \en{dataset} tidak terlalu membawa pengaruh. Di bawah jumlah \en{timestep} tertentu, durasi \en{training} versi \en{improved} tidak jauh berbeda dengan versi asli, sedangkan di atas jumlah \en{timestep} tertentu, versi \en{improved} secara konsisten mengurangi durasi \en{training} sebesar 2.5\% hingga 3.2\%.
