\chapter{Pendahuluan}

% Bab Pendahuluan secara umum yang dijadikan landasan kerja dan arah kerja penulis tugas akhir, berfungsi mengantar pembaca untuk membaca laporan tugas akhir secara keseluruhan.

\section{Latar Belakang}

% Latar Belakang berisi dasar pemikiran, kebutuhan atau alasan yang menjadi ide dari topik tugas akhir. Tujuan utamanya adalah untuk memberikan informasi secukupnya kepada pembaca agar memahami topik yang akan dibahas.  Saat menuliskan bagian ini, posisikan anda sebagai pembaca --- apakah anda tertarik untuk terus membaca?

Saat ini, \en{deep learning} semakin populer dan banyak digunakan oleh berbagai kalangan baik di dalam penelitian maupun industri. Salah satu alasannya adalah kinerja \en{deep learning} yang mampu menungguli metode-metode pembelajaran mesin lainnya dalam berbagai permasalahan AI seperti penglihatan komputer, pengenalan ucapan, dan translasi bahasa natural. Alasan lainnya, di masa kini ukuran \en{dataset} semakin membesar, dan \en{deep learning} dapat bekerja dengan baik dalam memproses \en{dataset} berukuran besar relatif terhadap metode-metode pembelajaran mesin lainnya.

Yang membedakan \en{deep learning} dari \en{shallow learning} itu sendiri adalah penggunaan \en{hidden layer} berjumlah banyak. Banyaknya jumlah \en{hidden layer} memungkinkan terdapatnya berbagai arsitektur \en{deep learning}, seperti \en{fully-connected neural network} (FNN), \en{convolutional neural network} (CNN), dan \en{recurrent neural network} (RNN).

Model \en{deep learning} pada umumnya membutuhkan banyak \en{layer} dan parameter, serta \en{dataset} berukuran besar untuk mendapatkan kinerja optimal. Selain itu, beberapa arsitektur seperti RNN cenderung menggunakan lebih banyak \en{layer} dan parameter dari arsitektur-arsitektur lainnya. Faktor-faktor ini menimbulkan beberapa permasalahan pada saat melatih model, seperti panjangnya durasi dan besarnya sumber daya komputasi yang dibutuhkan untuk melatih model.

Terdapat berbagai cara untuk menangani permasalahan-permasalahan tersebut. Pertama, karena sifatnya yang dapat dikomputasi secara paralel, pelatihan model \en{deep learning} sering menggunakan GPU untuk mempersingkat waktu yang dibutuhkan. Selain itu, untuk mengurangi penggunaan memori dan mempercepat konvergensi, \en{dataset} sering dibagi-bagi menjadi beberapa `\en{mini-batch}` sehingga jumlah data yang diproses pada suatu waktu dapat berkurang. Konvergensi pun menjadi lebih cepat karena \en{update} parameter yang lebih sering (tiap \en{mini-batch} ketimbang tiap seluruh \en{dataset} selesai diproses). Metode seperti demikian disebut \en{stochastic gradient descent} (SGD).

Akan tetapi, metode SGD memiliki \en{trade-off} tersendiri. Semakin kecil ukuran \en{batch}, \en{batch} tersebut menjadi kurang representatif terhadap data sebenarnya, sehingga dapat menimbulkan penurunan kinerja (karena terjebak di \en{local minima}). Memang, menggunakan ukuran \en{batch} yang besar tidak selalu meningkatkan kinerja, sehingga tetap harus dilakukan eksperimen untuk mencari ukuran \en{batch} paling optimal. Masalahnya, ukuran \en{batch} berukuran besar tidak selalu mungkin dengan sumber daya yang ada, karena tingginya kebutuhan memori, durasi pelatihan, dan lainnya.

Salah satu sumber daya yang dibutuhkan pada pelatihan model adalah memori. Melatih model \en{deep learning} membutuhkan semakin banyak memori seiring bertambahnya jumlah \en{layer}, parameter, dan \en{dataset}. Ditambah lagi, melatih model \en{deep learning} sering menggunakan GPU untuk meningkatkan kinerja karena kemampuannya untuk melakukan komputasi paralel. Sayangnya, kapasitas memori GPU cenderung lebih kecil dibandingkan dengan kapasitas memori CPU (RAM).

\section{Rumusan Masalah}

% Rumusan Masalah, berisi masalah utama yang dibahas dalam tugas akhir. Rumusan masalah yang baik memiliki struktur sebagai berikut:

% \begin{enumerate}
%     \item Penjelasan ringkas tentang kondisi/situasi yang ada sekarang terkait dengan topik utama yang dibahas Tugas Akhir.
%     \item Pokok persoalan dari kondisi/situasi yang ada, dapat dilihat dari kelemahan atau kekurangannya. \textbf{Bagian ini merupakan inti dari rumusan masalah.}
%     \item Elaborasi lebih lanjut yang menekankan pentingnya untuk menyelesaikan pokok persoalan tersebut.
%     \item Usulan singkat terkait dengan solusi yang ditawarkan untuk menyelesaikan persoalan.
% \end{enumerate}

% Penting untuk diperhatikan bahwa persoalan yang dideskripsikan pada subbab ini akan dipertanggungjawabkan di akhir pelaksanaan tugas akhir apakah terselesaikan atau tidak.

Seperti telah dijelaskan pada Latar Belakang, pelatihan model \en{deep learning} membutuhkan banyak sumber daya, salah satunya memori. Terlebih lagi, pelatihan \en{deep learning} seringkali menggunakan GPU yang, relatif terhadap CPU, memiliki memori jauh lebih sedikit. Salah satu cara mengurangi penggunaan memori adalah dengan mengurangi ukuran \en{batch}, namun cara ini dapat menurunkan kinerja model. Untuk mendukung pelatihan model dengan ukuran \en{batch} lebih besar, dibutuhkan optimisasi memori.

Salah satu \en{framework} \en{deep learning}, TensorFlow, mendukung optimisasi memori dengan \en{memory swapping}, yaitu memindahkan data dari memori GPU ke memori CPU apabila tidak terpakai dalam waktu dekat. Akan tetapi, teknik \en{memory swapping} pada TensorFlow relatif sederhana, yaitu melakukan \en{memory swapping} secara langsung ketika memori GPU tidak mencukupi. Oleh karena itu, pada tugas akhir ini penulis bertujuan mengoptimisasi \en{memory swapping} pada TensorFlow, khususnya pada model berbasis RNN seperti LSTM. Model berbasis RNN dipilih karena sifat pembangkitan \en{neural network}-nya yang dinamis.

\section{Tujuan}

% Tuliskan tujuan utama dan/atau tujuan detil yang akan dicapai dalam pelaksanaan tugas akhir. Fokuskan pada hasil akhir yang ingin diperoleh setelah tugas akhir diselesaikan, terkait dengan penyelesaian persoalan pada rumusan masalah. Penting untuk diperhatikan bahwa tujuan yang dideskripsikan pada subbab ini akan dipertanggungjawabkan di akhir pelaksanaan tugas akhir apakah tercapai atau tidak.

Secara garis besar, tugas akhir ini bertujuan mengoptimisasi penggunaan memori GPU pada pelatihan model \en{deep learning} berbasis RNN, yaitu mengurangi penggunaan memori GPU maksimum pada suatu waktu. Harapan tercapainya tujuan ini adalah dengan sumber daya yang sama, dapat dilatih model dengan ukuran \en{batch} lebih besar, dapat dilatih model dengan lebih banyak \en{layer} dan parameter, dapat dilatih lebih banyak model pada suatu waktu, dan dapat dilatih model-model yang sebelumnya tidak dapat dilatih karena kekurangan memori.

\section{Batasan Masalah}

% Tuliskan batasan-batasan yang diambil dalam pelaksanaan tugas akhir. Batasan ini dapat dihindari (tidak perlu ada) jika topik/judul tugas akhir dibuat cukup spesifik.

Tugas akhir ini memiliki batasan-batasan masalah seperti berikut.

\begin{enumerate}
    \item \en{Training} model dilakukan pada sebuah komputer dengan \en{single GPU} (tidak terdistribusi).
    \item Model \en{deep learning} yang utama digunakan adalah model berbasis RNN seperti LSTM, namun jenis model lain dapat digunakan untuk perbandingan atau eksperimen.
    \item Proses \en{deep learning} dijalankan pada TensorFlow, \en{framework} pembelajarn mesin yang akan dioptimisasi sebagai tujuan dari tugas akhir ini.
\end{enumerate}

\section{Metodologi}

% Tuliskan semua tahapan yang akan dilalui selama pelaksanaan tugas akhir. Tahapan ini spesifik untuk menyelesaikan persoalan tugas akhir. Tahapan studi literatur tidak perlu dituliskan karena ini adalah pekerjaan yang harus Anda lakukan selama proses pelaksanaan tugas akhir.

Tahapan pengerjaan tugas akhir ini dimulai dengan melakukan analisis dan merumuskan permasalahan yang akan dipecahkan. Selanjutnya, melakukan pencarian terhadap solusi yang tepat untuk memecahkan permasalahan tersebut, salah satunya dengan melakukan studi literatur. Setelah memutuskan solusi yang tepat, melakukan perancangan desain eksperimen serta implementasi yang akan digunakan untuk melakukan eksperimen. Langkah selanjutnya adalah mengimplementasi solusi dan bereksperimen sesuai rancangan yang dibuat sebelumnya. Setelahnya, menganalisis dan mengevaluasi hasil eksperimen hingga akhirnya menarik kesimpulan terhadap permasalahan dan solusi yang diajukan.

\section{Jadwal Pelaksanaan Tugas Akhir}

% Tuliskan rencana kegiatan dan jadwal (dirinci sampai per minggu) mulai dari awal pelaksanaan Tugas Akhir I s.d.\ sidang tugas akhir berikut \en{milestones} dan \en{deliverables} yang harus diberikan. Jadwal ini dapat dibantu dengan membuat sebuah tabel \en{timeline.}

TBD.
